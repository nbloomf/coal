\documentclass{ximera}

\title{Embedding google spreadsheets in a ximera document}
\author{Nathan Bloomfield}

\begin{abstract}
  wherein it is demonstrated how to do such of things of that
\end{abstract}

\begin{document}
\maketitle

Let's embed some Google Spreadsheets!

First we need a google spreadsheet with some data in it. How about \url[this one]{https://docs.google.com/spreadsheets/d/16yym0ge5YWl_7i7BBsosd6ji9PNab5DWKnQLVjxJRDo/edit?usp=sharing}. Before we can embed a spreadsheet it must be \emph{published to the web} (the option for this is under the File menu).

The command for embedding a spreadsheet is \texttt{googleSheet}. It has 5 required arguments (two of which can be empty) and may be used in one of three ways:
\begin{enumerate}
\item To embed an entire multi-page spreadsheet document, with ``tabs'' for navigating between pages;
\item To embed a single sheet, which does not include navigation tabs; and
\item To embed only a \emph{range} of cells from a single sheet
\end{enumerate}

\subsection{Embedding a whole spreadsheet}

The syntax is \verb|\googleSheet{docID}{width}{height}{}{}|, where \texttt{width} and \texttt{height} are the dimensions (in pixels) of the embedded spreadsheet on the page. If the spreadsheet is larger than this region, scrollbars will appear.

\googleSheet{16yym0ge5YWl_7i7BBsosd6ji9PNab5DWKnQLVjxJRDo}{400}{300}{}{}


\subsection{Embedding a single sheet}

\googleSheet{16yym0ge5YWl_7i7BBsosd6ji9PNab5DWKnQLVjxJRDo}{400}{300}{0}{}


\subsection{Embedding a range of cells from one sheet}

\googleSheet{16yym0ge5YWl_7i7BBsosd6ji9PNab5DWKnQLVjxJRDo}{400}{300}{0}{A2:A3}

\end{document}
