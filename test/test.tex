\documentclass{ximera}

\title{Embedding google spreadsheets in a ximera document}
\author{Nathan Bloomfield}

\begin{abstract}
  wherein it is demonstrated how to do such of things of that
\end{abstract}

\begin{document}
\maketitle

Let's embed some Google Spreadsheets! Note that these embedded spreadsheets will be \textbf{Read Only}; this is a limitation of Google spreadsheets (as far as I can tell). However, they do update live -- or with a few minutes' delay -- so you can edit the master spreadsheet and any embedded copies will (eventually) reflect the changes. That's nifty!

First we need a google spreadsheet with some data in it. How about \link[this one]{https://docs.google.com/spreadsheets/d/16yym0ge5YWl_7i7BBsosd6ji9PNab5DWKnQLVjxJRDo/edit?usp=sharing}. Before we can embed a spreadsheet it must be \textit{published to the web} (the option for this is under the File menu).

The command for embedding a spreadsheet is \texttt{googleSheet}. It has 5 required arguments (two of which can be empty) and may be used in one of three ways:
\begin{enumerate}
\item To embed an entire multi-page spreadsheet document, with ``tabs'' for navigating between pages;
\item To embed a single sheet, which does not include navigation tabs; and
\item To embed only a \textit{range} of cells from a single sheet
\end{enumerate}

\subsection{Embedding a whole spreadsheet}

The syntax for embedding an entire spreadsheet is
\begin{verbatim}
\googleSheet{document-ID}{width}{height}{}{}
\end{verbatim}
where \texttt{width} and \texttt{height} are the dimensions (in pixels) of the embedded spreadsheet on the page. If the spreadsheet is larger than this region, scrollbars will appear. \texttt{document-ID} is the unique identifier of the shared spreadsheet. To find this value, open your spreadsheet and select ``Publish to the web'' from the File menu. Go to the ``Embed'' tab and select ``Entire document'' in the drop-down menu. The text area now shows the embed URL; all we want is the document ID, which is the giant string of characters after \texttt{docs.google.com/spreadsheets/d/} and before the next slash.

For example, the spreadsheet we linked to above has the entire-document-embed-URL
\begin{verbatim}
https://docs.google.com/spreadsheets/d/16yym0ge5YWl_7i7BBsosd6ji9PNab5DWKnQLVjxJRDo/pubhtml?widget=true&amp;headers=false
\end{verbatim}
and the document ID is
\begin{verbatim}
16yym0ge5YWl_7i7BBsosd6ji9PNab5DWKnQLVjxJRDo
\end{verbatim}

Now to embed this spreadsheet. The command
\begin{verbatim}
\googleSheet{16yym0ge5YWl_7i7BBsosd6ji9PNab5DWKnQLVjxJRDo}{400}{300}{}{}
\end{verbatim}
Gives the following:

\googleSheet{16yym0ge5YWl_7i7BBsosd6ji9PNab5DWKnQLVjxJRDo}{400}{300}{}{}


\subsection{Embedding a single sheet}

The syntax for embedding a single sheet is
\begin{verbatim}
\googleSheet{document-ID}{width}{height}{sheet-ID}{}
\end{verbatim}
where \texttt{document-ID}, \texttt{width}, and \texttt{height} are as defined above. \texttt{sheet-ID} is the unique identifier of the sheet you want to embed. To find this identifier, again choose ``Publish to the web'' in the File menu and select the ``Embed'' tab. This time select the sheet you want to embed in the drop-down menu. The text area now shows the embed URL, and the sheet identifier is the value of the \texttt{gid} key in this URL.

For example, the embed url for Sheet 1 in our example document is 
\begin{verbatim}
https://docs.google.com/spreadsheets/d/16yym0ge5YWl_7i7BBsosd6ji9PNab5DWKnQLVjxJRDo/pubhtml?gid=0&amp;single=true&amp;widget=true&amp;headers=false
\end{verbatim}
and the sheet ID for Sheet 1 is
\begin{verbatim}
0
\end{verbatim}
Note that Sheet 1 always(?) has an ID of \texttt{0}, but subsequent IDs are not sequential. Sheet 2 in my example document has ID \texttt{2081259156}, because why not.

As an example, the command
\begin{verbatim}
\googleSheet{16yym0ge5YWl_7i7BBsosd6ji9PNab5DWKnQLVjxJRDo}{400}{300}{0}{}
\end{verbatim}
produces the following:

\googleSheet{16yym0ge5YWl_7i7BBsosd6ji9PNab5DWKnQLVjxJRDo}{400}{300}{0}{}


\subsection{Embedding a range of cells from one sheet}

The syntax for embedding a range of cells is
\begin{verbatim}
\googleSheet{document-ID}{width}{height}{sheet-ID}{range-selector}
\end{verbatim}
where the only new argument, \texttt{range-selector}, is a pair of cell names separated by a colon.

For example, the command
\begin{verbatim}
\googleSheet{16yym0ge5YWl_7i7BBsosd6ji9PNab5DWKnQLVjxJRDo}{400}{300}{0}{A2:A3}
\end{verbatim}
produces the following:

\googleSheet{16yym0ge5YWl_7i7BBsosd6ji9PNab5DWKnQLVjxJRDo}{400}{300}{0}{A2:A3}

\end{document}
